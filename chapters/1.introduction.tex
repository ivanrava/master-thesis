In cui si descrive a grandi linee il problema affrontato nel seguito.

\section{Contesto}
L'informatica e le reti di telecomunicazione sono oggi componenti fondamentali e pervasive delle nostre vite quotidiane. Dispositivi di ogni genere e reti di ogni dimensione sono diventati parte integrante della nostra quotidianità, spesso senza che ce ne rendiamo conto.

In questo scenario, garantire la sicurezza di tali infrastrutture assume un'importanza crescente. Ogni nodo di rete, anche il più apparentemente insignificante, può rappresentare un valore strategico nelle mani di un attaccante esperto, sia come obiettivo diretto sia come punto di ingresso per attacchi più estesi.

Questi rischi interessano chiunque disponga di servizi esposti in rete, indipendentemente dalla loro popolarità. La rete è periodicamente scandagliata da attaccanti che utilizzano strumenti automatizzati per individuare servizi con vulnerabilità note, cercando di sfruttarle per ottenere un vantaggio immediato. La necessità di verificare regolarmente la sicurezza delle proprie risorse è ancor più critica per grandi aziende e organizzazioni altamente esposte.

Per queste ragioni, effettuare controlli regolari sulla sicurezza della propria rete e dei servizi esposti su Internet è una pratica essenziale.

\section{Verifica della sicurezza di un nodo di rete}
La sicurezza dei nodi di rete può essere verificata tramite procedure comunemente note come \textbf{vulnerability assessment} (lett. \emph{valutazione delle vulnerabilità}). L'obiettivo di queste procedure è:
\begin{itemize}
    \item Identificare eventuali vulnerabilità nei servizi visibili dalla rete.
    \item Verificare se tali vulnerabilità sono associate a rischi di sicurezza noti.
    \item Classificare e valutare il livello di rischio delle vulnerabilità, stabilendo una gerarchia di priorità per la loro risoluzione e fornendo indicazioni utili agli amministratori di sistema.
\end{itemize}

\subsection{Strumenti disponibili}
Esistono molte modalità per eseguire queste procedure, che possono essere più o meno manuali. In ambito produttivo è comune l'uso di \textbf{vulnerability scanner}, applicativi dedicati che analizzano la sicurezza di una lista di \emph{host} di rete, confrontando le vulnerabilità rilevate con \emph{database} costantemente aggiornati.

Gli scanner di vulnerabilità possono essere gratuiti o a pagamento, con diverse licenze, caratteristiche e requisiti. Nonostante le differenze, condividono lo stesso compito fondamentale: individuare vulnerabilità e confrontarle con i database noti. Le principali differenze tra questi strumenti riguardano il costo e le funzionalità aggiuntive, come reportistica avanzata, suggerimenti per la mitigazione, correzioni automatiche, analisi euristiche o altre caratteristiche che possono semplificare il lavoro dell'utente.

\section{Casi d'uso}
La sicurezza della propria presenza in rete rappresenta un'esigenza fondamentale per chiunque, indipendentemente dalla specificità dei requisiti.

Tuttavia, l'installazione e configurazione degli scanner menzionati in precedenza risultano spesso proibitive per la maggior parte degli utenti per diverse ragioni:
\begin{itemize}
    \item Mancanza di risorse economiche necessarie per acquistare la licenza dell'applicativo o per sostenere i costi dell'hardware richiesto.
    \item Insufficienza delle competenze tecniche per installare e configurare correttamente lo scanner.
    \item Difficoltà nell'interpretare i risultati ottenuti, ad esempio nel distinguere tra vulnerabilità che possono essere ignorate in sicurezza e quelle che richiedono interventi immediati.
\end{itemize}

In questo contesto, per \emph{penetration testing} si intende l'attività di simulazione di attacchi informatici condotta da esperti per identificare vulnerabilità sfruttabili in un sistema, rete o applicazione. A differenza del solo \emph{vulnerability assessment}, che si limita a individuare e classificare le falle di sicurezza tramite strumenti automatici, il \emph{penetration testing} prevede un'analisi più approfondita e manuale, in cui il tester assume il ruolo di un potenziale attaccante.

In aggiunta, l'esecuzione di una scansione di sicurezza può comportare rischi intrinseci. Alcuni servizi di rete particolarmente obsoleti o configurati in modo inadeguato potrebbero reagire negativamente alla natura invasiva della scansione, causando interruzioni o malfunzionamenti. Per questa ragione, si raccomanda di affidare comunque tali attività a personale qualificato.

\subsection{Vulnerability Assessment e Penetration Testing}
Il termine VAPT combina due tipi di attività\footnote{Ciò detto, a volte alcune aziende per VAPT intendono erroneamente la sola attività di \emph{vulnerability assessment}.}:
\begin{itemize}
    \item Il \textbf{Vulnerability Assessment}, che consiste nell'utilizzo di scanner automatici per identificare vulnerabilità e associarle a rischi noti.
    \item Il \textbf{Penetration Testing}, che prevede una simulazione manuale di attacchi condotti da esperti per individuare e sfruttare debolezze nei sistemi. Questo approccio consente di testare non solo la presenza di vulnerabilità, ma anche la capacità di risposta dell'organizzazione a potenziali attacchi.
\end{itemize}

Durante il \emph{penetration testing}, i tester assumono il ruolo di attaccanti e tentano di sfruttare falle note, errori di configurazione o debolezze specifiche nel software. I risultati di questa attività vengono documentati in un report dettagliato che include:
\begin{itemize}
    \item Descrizione delle vulnerabilità riscontrate, con relativa classificazione del rischio.
    \item Dimostrazione degli exploit utilizzati per evidenziare le debolezze.
    \item Raccomandazioni mirate per mitigare i rischi e rafforzare la sicurezza complessiva del sistema.
\end{itemize}

\section{Il problema}
Nonostante l'importanza delle attività di VAPT, i costi e le complessità associate ai moderni strumenti di sicurezza le rendono spesso difficili da adottare per molte organizzazioni. In particolare, utenti con esigenze meno stringenti potrebbero necessitare di controlli di sicurezza più semplici, eseguiti con frequenza ridotta (ad esempio, su base mensile o addirittura \emph{una tantum}).

Per rispondere a queste esigenze, l'obiettivo di questo lavoro è integrare uno scanner di sicurezza esistente con un'interfaccia semplificata che ne mantenga l'efficacia, ma che sia accessibile anche per utenti non esperti.

In questa tesi si analizzano, progettano e sviluppano le soluzioni software necessarie per realizzare tale sistema, includendo:
\begin{itemize}
    \item Analisi dei requisiti e progettazione del sistema.
    \item Sviluppo e implementazione della soluzione proposta.
    \item Procedure di installazione e configurazione.
    \item Discussione sui possibili sviluppi futuri e sull'estendibilità del sistema.
\end{itemize}

L'obiettivo finale è produrre non solo un prototipo funzionale, ma un prodotto utilizzabile in contesti reali, caratterizzato da semplicità d'uso, solidità architetturale e flessibilità per integrazioni future.

\section{Terminologia specifica}
Di seguito vengono definiti alcuni termini tecnici utilizzati nel contesto della sicurezza informatica:

\begin{itemize}
    \item \textbf{EOL (End of Life)}: indica la fine del supporto tecnico e degli aggiornamenti di sicurezza per un software o sistema operativo. Dopo questa fase, il sistema diventa vulnerabile a nuove minacce.
    \item \textbf{DoS (Denial of Service)}: attacco volto a rendere un sistema o servizio non disponibile per gli utenti legittimi tramite sovraccarico o interruzioni.
    \item \textbf{RCE (Remote Code Execution)}: vulnerabilità che consente l'esecuzione remota di codice arbitrario su un sistema, una delle minacce più gravi per la sicurezza.
    \item \textbf{Privilege Escalation}: tecnica che consente a un attaccante di ottenere privilegi di accesso più elevati su un sistema, spesso sfruttando vulnerabilità note.
    \item \textbf{Buffer Overflow}: vulnerabilità in cui dati eccedenti la capacità del buffer di memoria possono causare l'esecuzione di codice arbitrario, spesso portando a RCE.
    \item \textbf{Fingerprinting}: tecnica per identificare e caratterizzare un sistema, servizio o applicazione raccogliendo informazioni sulla configurazione e sulle caratteristiche, spesso utilizzata analizzando pacchetti di rete e comportamento dei protocolli.
\end{itemize}