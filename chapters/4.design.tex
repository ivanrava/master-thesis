In cui si descrive la progettazione del software a un più basso livello: scelte progettuali, tecnologie e linguaggi adottati.

\section{Greenbone OpenVAS}
Come backend di scansione per il sistema, inizialmente si è optato per interfacciarsi a \textbf{Greenbone OpenVAS}.

\section{Base dati}
Il framework di Greenbone già include un suo database interno per la gestione degli utenti e dei ruoli associati ad essi, così come per le scansioni, i risultati e la reportistica. Inoltre, gran parte della logica di business rilevante è già definita e può essere personalizzata con un sistema di permessi abbastanza preciso e granulare, rendendo possibile adattarla facilmente ai nostri scopi.

Per questo motivo, si è deciso di non introdurre ridondanza e complessità con un ulteriore database, preferendo invece rimanere aderenti alla base di dati esistente, effettivamente sposando in tutto e per tutto la struttura dei dati proposta da Greenbone.

\subsection{Schema dei dati}
Lo schema dei dati proposto da Greenbone è 

\section{Gerarchia degli utenti}
Si è scelto per la gerarchia a tre livelli \ref{3-level}, lasciando la gerarchia a quattro livelli \ref{4-level} come futuro sviluppo, essendo questa una semplice estensione della prima possibilità.