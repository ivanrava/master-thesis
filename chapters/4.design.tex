In cui si descrive la progettazione del software a un più basso livello: scelte progettuali, tecnologie e linguaggi adottati.

\section{Greenbone OpenVAS}
Come backend di scansione per il sistema, inizialmente si è optato per interfacciarsi a \textbf{Greenbone OpenVAS}.

\section{Gerarchia degli utenti}
Si è scelto per la gerarchia a tre livelli \ref{3-level}, lasciando la gerarchia a quattro livelli \ref{4-level} come futuro sviluppo, essendo questa una semplice estensione della prima possibilità.

\section{Database}
Il framework di Greenbone già include un suo database per la gestione degli utenti e di ruoli associati, così come per le scansioni, i risultati e la reportistica. Inoltre, gran parte della logica di business è già definita e può essere personalizzata con un sistema di permessi abbastanza preciso e granulare.

Per questo motivo, si è deciso di non introdurre ridondanza e complessità con un ulteriore database per limitarsi l'istanza di PostgreSQL integrata in GVM, preferendo rimanere aderenti alla base di dati esistente, effettivamente sposando in tutto e per tutto la struttura dei dati nel framework Greenbone.

