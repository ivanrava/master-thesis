In cui si spiega brevemente il processo di installazione del software su di un server di produzione.

\section{Struttura}
Il sistema così realizzato finora consta di due componenti:
\begin{itemize}
    \item \textbf{SPA / frontend applicativo}: interagisce con il backend tramite protocollo HTTP.
    \item \textbf{REST API / backend applicativo}: risponde alle richieste del frontend attraverso l'interfacciamento con la socket UNIX del sistema Greenbone.
\end{itemize}
Dall'ultimo punto segue che la REST API e l'istanza del sistema Greenbone devono trovarsi necessariamente sullo stesso server. In alternativa, almeno il demone GVM deve essere presente sullo stesso server.

La SPA di frontend tuttavia può trovarsi anche su di un server differente e interagire con l'API del backend Flask tramite protocollo HTTP. Questo può aprire interessanti possibilità di ottimizzazione se si vuole usare delle CDN o altri strumenti di edge computing.

Nel nostro caso specifico tuttavia entrambi i componenti sono stati installati sullo stesso server, per pura semplicità.

\section{Containerizzazione}
L'installazione delle componenti e la loro integrazione con il sistema Greenbone sono state entrambe realizzate attraverso Docker (e più nello specifico, Docker Compose).

Infatti, come già detto in \ref{greenbone-community-containers}, Greenbone mette a disposizione un'installazione ufficiale e supportata dell'intero framework attraverso la piattaforma di Docker Compose. Per un ambiente di produzione questa strategia garantisce riproducibilità e affidabilità, ma anche un aggiornamento in generale estremamente semplice, per il quale basta distruggere i container esistenti, aggiornare le immagini di Docker e ricreare i container.

Questo di fatto guida il processo di installazione verso una strategia semplice, ma efficace: dockerizzare entrambe le componenti realizzate per questo progetto (frontend e backend) e interconnetterle in Docker Compose tramite due container appositi.

\subsection{Dockerizzazione - Backend}
Il backend è stato dockerizzato secondo il seguente procedimento, seguendo a grandi linee il \texttt{Dockerfile}:

\subsection{Dockerizzazione - Frontend}

\subsection{Interconnessione in Docker Compose}
